%\documentclass[11pt,a4paper,twoside]{book}
\documentclass[11pt,a4paper,oneside]{book}
 %\documentclass[oneside]{book}
% Misc
\usepackage[nottoc]{tocbibind} % Bibliography in toc
\usepackage{makeidx} \usepackage{graphicx} \usepackage{color}
\usepackage{parskip} \usepackage{natbib}
\definecolor{hhblue}{RGB}{0,73,133}

% Fancy headers
\usepackage{fancyhdr} \lhead{\slshape \nouppercase{\rightmark}}
\rhead{\slshape \nouppercase{\leftmark}}

% AMS Packages
\usepackage[intlimits]{amsmath} \usepackage{amscd}
\usepackage{amsxtra} \usepackage{amssymb}

% Hyperlinks
\usepackage{hyperref}
\hypersetup{colorlinks=true,linkcolor=hhblue,citecolor=hhblue,urlcolor=hhblue}

% Language
\usepackage[T1]{fontenc} \usepackage[swedish]{babel}
\usepackage[utf8]{inputenc}

%\iffalse
\usepackage{titlesec}
\titleformat{\chapter}[display]
  {\normalfont\huge\bfseries}{}{0pt}{\Huge}
  %{\normalfont\huge\bfseries}{}{0pt}{\Huge}
  {\normalfont\huge\bfseries\color{hhblue}}
\titlespacing*{\chapter}
  {0pt}{10pt}{30pt} % changed
%\fi
% Font
\usepackage[sc]{mathpazo} \usepackage{palatino}

% Page setup
\unitlength=1mm \oddsidemargin 0.0cm \evensidemargin 0.0cm \topskip
0.0cm \topmargin -0.5cm
% \headheight 0.0cm \headsep 0.0cm
\textheight 24.0cm \textwidth 16cm
\let\cleardoublepage\clearpage
\setcitestyle{super}


\usepackage{float}
\restylefloat{table}

\setlength{\arraycolsep}{1.4pt}


\begin{document}
\pagestyle{empty}

\frontmatter

% Framsida---------------------------------------------------------------------
\begin{titlepage}
  \begin{center}
  \end{center}
  \vspace{3cm}
  \begin{center}
    \hrule \vspace{0.5cm}
    {\Huge \bfseries \sffamily \color{hhblue} Projektplan(Raviolimaskin)}\\
   % \vspace{0.5cm} {\Large\emph{Kanske behövs någon undertext också}}
    \vspace{0.8cm} \hrule \vspace{2cm} {\Large{Reshad Ahmadi , Maryam Bayat}}\\
   %\hspace{1.5cm}
    %{\Large{Kalle Anka}}\\
    \vspace{2cm}
    \today\\
    \vspace{3cm}
    Examensarbete (Raviolimaskin)\\
    \vspace{1.5cm}
    Handledare: Kenneth Nilsson\\
   %\hspace{1.5cm}
    %{Tommy Salomonsson}\\ 
    \vspace{0.5cm} Examinator: Björn Åstrand \vfill
    \includegraphics[width=4cm]{images/hh_logo.jpg}\\
    HÖGSKOLAN I HALMSTAD\\
    Sektionen för Informationsvetenskap, \\
    Data- och Elektroteknik
  \end{center}
\end{titlepage}

% Sammanfattning----------------------------------------------------------------
%\chapter{Sammanfattning}
%\chapter{Abstract}
%Here is the nice abstract written in english.
% Förord-----------------------------------------------------------------------
%\chapter{Förord}
%\pagestyle{fancy}
% Innehållsförteckning---------------------------------------------------------
%\tableofcontents
%\mainmatter
%Here is the nice abstract written in english.
% Förord-----------------------------------------------------------------------
%\chapter{Förord}
%\pagestyle{fancy}


% Innehållsförteckning---------------------------------------------------------
\tableofcontents
\mainmatter

\chapter{Inledning}
Detta projekt ämnat till att skapa en Ravioli maskin. Raviolin är en traditionell italiensk maträtt bestående av rundor eller kvadratiska pastadeg med fyllning \cite{engproc}. Fyllningen kan bestå av till example köttfärs, skinka och ost. Raviolin serveras ofta i en tomatsås eller köttfärssås. vegetarisk ravioli kan exemplvis fyllas med purjiolök och spenat.\\

Att laga Ravioli hemma har varit jobbigt och tidskrävande. Det tar för mycket tid att fylla på en ravioli deg(utkavlade degen) och resultaten inte blir likadan för alla kuddar.\\

Det finns olika typer av Ravioli maskiner på markanden just nu. En typ av ravioli maskin som visas på figur1, underlättar processen men det mesta görs manuellt.\\

	 		 		\begin{figure}[h]
	 		 			\begin{center}
	 		 				\includegraphics[scale=0.5]{images/raviolimoldwithfilling.jpg}
	 		 				\caption{Raviolimaskin}
	 		 				\label{ravioli}	
	 		 			\end{center}
	 		 		\end{figure}
Den andra typen av maskinen är väldigt stor och priset är högt som medför att de inte kan användas av hushåll, se figur 3. Den typen finns färdig på marknaden.
 		\begin{figure}[h]
 			\begin{center}
 				\includegraphics[scale=2]{images/pastamachine.jpg}
 				\caption{Industriell Pasta/Raviolimaskin}
 				\label{pastamaskin}	
 			\end{center}
 		\end{figure}

Idén bakom projektet baseras på behov av en Ravioli maskin och potentiell marknad för den. Tanken är att man utvecklar en liten och bilig Ravioli maskin som kan vara användbar hemma.		

\section{Syfte och mål} % Goals
Detta projekt syftar på att utveckla en Raviolimaskin som ska fylla i Raviolidegen med ifyllningsmaterial och tillsluta degen automatiskt. Maskinen ska vara anpassad till hushåll i storlek, pris och användbarhet.

Följande krav har ställts på maskinen:

\begin{itemize}
	\item Maskinen ska automatiskt applicera fyllningsmaterialet på Raviolideg.
	\item Användaren ska kunna ha vilken fyllningsmaterial som helst för att fylla på raviolin.
	\item Maskinen ska automatisk tillsluta degen.
	\item Raviolideg ska placeras manuellt på maskinens degform.
	\item Raviolin ska plockas bort manuellt ur maskinen.
	\item Maskinen ska kommunicera med användaren via en display.
	\item Det ska vara lätt att rengöra maskinen.	
\end{itemize}

\section{Avgränsningar} % Constraints
Eftersom tiden är låst till en deadline som inte kan flyttas och personalresurer är begränsande,
kommer vi inte ha maskinen i metall.

En avgränsning ska vara att alla maskinens delar kommer att konstrueras med använndning av 3D-skrivare och plast som material. I slutet av projektet ska en plastmodel av maskinen utvecklas. Detta För att det är tar rätt mycket tid och resurser om man vill konstruera maskinen med t.ex. stål.


%-----------------------------------------------------------

\chapter{Teori}
\section{Existerande Raviolimaskiner}
De Raviolimaskiner som finns på marknad innehåller två huvuddelar, en pump för fyllning och en motordriven degform. 

Ett exempel på en Raviolimaskin visas på Figuren~\ref{raviolihemma}. Den består av två cylindriska degformar och en lucka där man fyller maskinen med fyllningsmaterial. Maskinens degfomar fungerar även som pump genom att de drar in fyllningsmaterialet när man snurrar dem m.h.a. ett handtag eller en motor.
 	\begin{figure}[h]
 		\begin{center}
 			\includegraphics[scale=0.4]{images/ravioli_machine_comment.jpg}
 			\caption{Raviolimaskin bestående av två cylindriska degformar~\cite{raviolimaskinbutik} }
 			\label{raviolihemma}	
 		\end{center}
 	\end{figure}
 
Den industeriella Raviolimaskinen som figur~\ref{pastamaskin} visar, fungerar med samma princip som maskinen på figur~\ref{raviolihemma}. Denna maskin gör allt från placering av en Raviolideg på maskinens degform till den slutar degen automatiskt.

Ett annat exempel på en industriell Raviolimaskin visas på figur ~\ref{industraviol_2}. Denna maskin består av en pump, en degform och en rullbana. Den typen av maskin fyller en Ravioli i tag och placering av Raviolideg på maskinen degform görs manuellt.
   .
 \begin{figure}[ht]
 	\begin{center}
 		\includegraphics[scale=0.4]{images/industriell_machine_comment.jpg}
 		\caption{Industriell Raviolimaskin som gör en Ravioli i tag(ref)}
 		\label{industraviol_2}	
 	\end{center}
 \end{figure}
 
 

 
Gruppmedlemmar har undersökt olika typer av pumpar för fyllning och olika sätt som degformen kan drivas med motor. Det finns olika modeller av pumpar och två av dem är Kolvdriven och kugghjul pump. För att driva degformen med motor analyserades två typer av motorer, likströmsmotor och stegmotor. De två typer av motorer lämpar detta projekt p.g.a. de är lätt att styra och tidigare erfarenhet att använda dem i ett projekt.

En undersökning gjordes på hur man kan detektera när degformen har pressat nog Raviolidegen för att tillsluta det med tillämpning av likströmsmotor eller stegmotor.



\chapter{Metod}
\section{Kunskapsläge}
Raviolimaskinen ska bestå av några mekaniska delar. För att designa maskinens delar ska CAD-tekniker användas. Att använda CAD-tekniker hjälper att rita maskinens olika delar och se resultatet innan man börjar konstruera dem fysiskt. Genom att designa med ett CAD-program, har man också möjligheten att analysera hållfasthet av maskinens delar genom att påverka virtuell kraft på dem. 


En mekanisk del av maskinen är en degform. Figuren ~\ref{degfrom} visar en dagform som används för att knyta Raviolidegen manuellt genom att trycka på formens sidor. För detta projekt har planerats att en eller två motorer ska trycka degformens sidor. Olika tekniker för att överföra motorers rörelseenergi till Raviolimaskinens degform ska undersökas. 

\begin{figure}[h]
	\begin{center}
		\includegraphics[scale=0.08] {images/degform.jpg}
		\caption{Degform för manuell ifyllning}
		\label{degfrom}	
	\end{center}
\end{figure}

En annan teknik som ska användas på detta projekt är regleringsteknik. Reglering kommer vara användbar när det gäller att reglera t.ex. den strömmen som går till elektroniska komponenter. Regleringsteknik kan också användas för eventuella systemidentifiering. 

Mikrokontrollern som ska användas för detta projekt är en Arduino Due. Programmeringsspråket ska vara C, men det är tänkt att använda Arduino IDE i fall man inte hinner programmera med C.






\section{Hur uppgifterna specifieras}
Uppgifterna specificerar genom att dela upp projektet i tre stora delar. Det första delen är mekanik som består av maskinens formgivning och analys av alla krafter som kommer påverkas på varje del. Kravet på mekaniken specificerar genom att varje del av maskinen ska orka bära de krafter som kommer påverka det.

Vidare ska finnas elektronikdel som består av en krets för att strömförsörja motorer och några eventuella sensorer. Krav på elektroniken kan specificera genom att alla komponenter(motorer och eventuella sensorer) får tillräcklig ström för att fungera rätt. Det är också tänkt att utveckla strömregulator som ska reglera strömmen som går till de elektroniska komponenter. 

Programmeringsdel av projektet tar hand om timingen på ett sätt att olika komponenter fungerar rätt och i rätt tid. Programmerings uppgifter omfattas att läsa av sensorers värde och driva motorer. 


\section{Metodbeskrivning}
Raviolimaskinens delar kommer konstrueras med hjälp av en 3D-skrivare. Detta mest för att det blir mycket lättare att skapa vissa delar som är svårt om man vill bilda med metall. Det blir också billigare att printa delar med plast än bygga dem med t.ex. stål. Resursbehovet för att printa alla maskinens delar är självklart tillgång till en 3D-skrivare, 4 dagar i vecka för en månad.

Projektets elektronik kommer utvecklas med användning av några elektroniska komponenter. En prototyp av kretsen ska utvecklas och användas under projektet. Ett kretskort ska tillverkas när prototypen har fungerat som det ska. Resursbehovet för att tillverka ett kretskort ska vara tillgång till skolans elverkstad. 
Eftersom det är en egen ide, är det tänkt att använda skolan resurser liksom elverkstad eller 3D-skrivare. Nödvändiga Komponenter t.ex. ABS-filament till 3D-printern eller elektroniska komponenter ska skaffas av projektets deltagare.


\section{Analys av resultat}
Projektet kan delas i Mekanik, elektronik och programmering. Mekanik testen genomförs på olika delar för att testa om de verkligen orkar bära de krafter som kommer påverka dem. Det finns mätutrustning liksom dynamometer för att mäta den kraften som en del måste kunna tåla. Man kan för testskul påverka lika mycket kraft på samma del för att kontrollera om den verkligen lyckas tåla den kraften.

Elektroniken kan analyseras genom att alla komponenter får den strömmen som är bestämt för dem. T.ex. om en motor skulle har fått 250 mA, kan man kontrollera att den försörjas med den bestämda strömmen under olika testfall och olika last på den.

För programmeringsdel kan testprogram skrivas som testar olika funktioner. Ett exempel för test av koden kan vara att kontrollera avläsning av en specifik sensor under en viss period och kontrollera resultatet. Mer specifik testfall för olika delar av Raniolinaskinen kommer speciferas när man har fått en tydligre bild av maskinen och dess olika delar.
\chapter{Tidsplan}
Examenarbetet krävs 20 timmar arbetsinsats i veckan. Därför måste läggas minst 350 timmar 
för att kunna klara arbetet. Följande är en grovplanering till projektet med tanke på de olika uppgifter som ska göras. \\


\begin{figure}[h]
	\begin{center}
		\includegraphics[scale=0.5] {images/tidsplan.png}
		\label{Tidsplan}	
	\end{center}
\end{figure}



%\begin{itemize}
% \item Krav nr 1: Bygga en reglercentral som ska ersätta innedelen.
 %\item Krav nr 2: Reglercentralen ska kommunicera med utedelen enbart med RS-485 via kontakter F1 och F2.
% \item Krav nr 3: Reglercentralen ska hantera två insignaler. En analog signal mellan 0 till 10 volt och en digital signal.
% \item Krav nr 4: Reglercentralen ska översätta kompressorns frekvens så att tex 1 volt analog insignal motsvarar 10\% av kompressorn kapacitet, 2 volt 20 \% osv.
% \item Krav nr 5: Värme eller kylläge på utedelen ska väljas med den digitala insignalen till centralen.\\*
%\end{itemize}

\backmatter
% Referenser--------------------------------------------------------------------
 \bibliographystyle{stylename}
\begin{thebibliography}{1}
\bibitem{engproc}
\href{http://www.wisegeek.com/what-is-ravioli.htm}{http://www.wisegeek.com/what-is-ravioli.htm}, engproc
\end{thebibliography}

\end{document}
