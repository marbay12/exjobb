%\documentclass[11pt,a4paper,twoside]{book}
\documentclass[11pt,a4paper,oneside]{book}
 %\documentclass[oneside]{book}
% Misc
\usepackage[nottoc]{tocbibind} % Bibliography in toc
\usepackage{makeidx} \usepackage{graphicx} \usepackage{color}
\usepackage{parskip} \usepackage{natbib}
\definecolor{hhblue}{RGB}{0,73,133}

% Fancy headers
\usepackage{fancyhdr} \lhead{\slshape \nouppercase{\rightmark}}
\rhead{\slshape \nouppercase{\leftmark}}

% AMS Packages
\usepackage[intlimits]{amsmath} \usepackage{amscd}
\usepackage{amsxtra} \usepackage{amssymb}

% Hyperlinks
\usepackage{hyperref}
\hypersetup{colorlinks=true,linkcolor=hhblue,citecolor=hhblue,urlcolor=hhblue}

% Language
\usepackage[T1]{fontenc} \usepackage[swedish]{babel}
\usepackage[utf8]{inputenc}

%\iffalse
\usepackage{titlesec}
\titleformat{\chapter}[display]
  {\normalfont\huge\bfseries}{}{0pt}{\Huge}
  %{\normalfont\huge\bfseries}{}{0pt}{\Huge}
  {\normalfont\huge\bfseries\color{hhblue}}
\titlespacing*{\chapter}
  {0pt}{10pt}{30pt} % changed
%\fi
% Font
\usepackage[sc]{mathpazo} \usepackage{palatino}

% Page setup
\unitlength=1mm \oddsidemargin 0.0cm \evensidemargin 0.0cm \topskip
0.0cm \topmargin -0.5cm
% \headheight 0.0cm \headsep 0.0cm
\textheight 24.0cm \textwidth 16cm
\let\cleardoublepage\clearpage
\setcitestyle{super}


\usepackage{float}
\restylefloat{table}

\setlength{\arraycolsep}{1.4pt}


\begin{document}
\pagestyle{empty}

\frontmatter

% Framsida---------------------------------------------------------------------
\begin{titlepage}
  \begin{center}
  \end{center}
  \vspace{3cm}
  \begin{center}
    \hrule \vspace{0.5cm}
    {\Huge \bfseries \sffamily \color{hhblue} Projektplan(Raviolimaskin)}\\
   % \vspace{0.5cm} {\Large\emph{Kanske behövs någon undertext också}}
    \vspace{0.8cm} \hrule \vspace{2cm} {\Large{Reshad Ahmadi , Maryam Bayat}}\\
   %\hspace{1.5cm}
    %{\Large{Kalle Anka}}\\
    \vspace{2cm}
    \today\\
    \vspace{3cm}
    Examensarbete (Raviolimaskin)\\
    \vspace{1.5cm}
    Handledare: Kenneth Nilsson\\
   %\hspace{1.5cm}
    %{Tommy Salomonsson}\\ 
    \vspace{0.5cm} Examinator: Björn Åstrand \vfill
    \includegraphics[width=4cm]{images/hh_logo.jpg}\\
    HÖGSKOLAN I HALMSTAD\\
    Sektionen för Informationsvetenskap, \\
    Data- och Elektroteknik
  \end{center}
\end{titlepage}

% Sammanfattning----------------------------------------------------------------
%\chapter{Sammanfattning}
%\chapter{Abstract}
%Here is the nice abstract written in english.
% Förord-----------------------------------------------------------------------
%\chapter{Förord}
%\pagestyle{fancy}
% Innehållsförteckning---------------------------------------------------------
%\tableofcontents
%\mainmatter
%Here is the nice abstract written in english.
% Förord-----------------------------------------------------------------------
%\chapter{Förord}
%\pagestyle{fancy}


% Innehållsförteckning---------------------------------------------------------
\tableofcontents
\mainmatter

\chapter{Introduktion}
Detta projekt ämnat till att skapa en Ravioli maskin. Raviolin är en traditionell italiensk maträtt bestående av rundor eller kvadratiska pastadeg med fyllning \cite{engproc}. Fyllningen kan bestå av till example köttfärs, skinka och ost. Raviolin serveras ofta i en tomatsås eller köttfärssås. vegetarisk ravioli kan exemplvis fyllas med purjiolök och spenat.\\

Att laga Ravioli hemma har varit jobbigt och tidskrävande. Det tar för mycket tid att fylla på en ravioli deg(utkavlade degen) och resultaten inte blir likadan för alla kuddar.\\

Det finns olika typer av Ravioli maskiner på markanden just nu. En typ av ravioli maskin som visas på figur1, underlättar processen men det mesta görs manuellt.\\

	 		 		\begin{figure}[h]
	 		 			\begin{center}
	 		 				\includegraphics[scale=0.5]{images/raviolimoldwithfilling.jpg}
	 		 				\caption{Raviolimaskin}
	 		 				\label{ravioli}	
	 		 			\end{center}
	 		 		\end{figure}
Den andra typen av maskinen är väldigt stor och priset är högt som medför att de inte kan användas av hushåll, se figur 3. Den typen finns färdig på marknaden.
 		\begin{figure}[h]
 			\begin{center}
 				\includegraphics[scale=2]{images/pastamachine.jpg}
 				\caption{Industriell Pasta/Raviolimaskin}
 				\label{pastamaskin}	
 			\end{center}
 		\end{figure}

Idén bakom projektet baseras på behov av en Ravioli maskin och potentiell marknad för den. Tanken är att man utvecklar en liten och bilig Ravioli maskin som kan vara användbar hemma.		

\section{Syfte och mål} % Goals
Detta projekt syftar på att utveckla en Raviolimaskin som ska fylla i Raviolidegen med ifyllningsmaterial och tillsluta degen automatiskt. Maskinen ska vara anpassad till hushåll i storlek, pris och användbarhet.

Följande krav har ställts på maskinen:

\begin{itemize}
	\item Maskinen ska automatiskt applicera fyllningsmaterialet på Raviolideg.
	\item Användaren ska kunna ha vilken fyllningsmaterial som helst för att fylla på raviolin.
	\item Maskinen ska automatisk tillsluta degen.
	\item Raviolideg ska placeras manuellt på maskinens degform.
	\item Raviolin ska plockas bort manuellt ur maskinen.
	\item Maskinen ska kommunicera med användaren via en display.
	\item Det ska vara lätt att rengöra maskinen.	
\end{itemize}

\section{Avgränsningar}
Vi avgränsar oss till kommunikation med en utedel, och den analoga signalen ska vara mellan 0-10 Volt.\\

\section{Begränsningar} % Constraints
Eftersom existerande verktyg används är den enda stora begränsningen den tid det tar att genomföra 
projektet. Tiden är låst till en deadline som inte kan flyttas, och personalresurser är begränsade. 
Följaktligen är kvaliteten den enda variabel som kan ändras om projektet löper risk att inte bli klar 
på utsatt tid.
%-----------------------------------------------------------
\newpage
\chapter{Metod}
\section{Kunskapsläge}
Raviolimaskinen ska bestå av några mekaniska delar. För att designa maskinens delar ska CAD-tekniker användas. Att använda CAD-tekniker hjälper att rita maskinens olika delar och se resultatet innan man börjar konstruera dem fysiskt. Genom att designa med ett CAD-program, har man också möjligheten att analysera hållfasthet av maskinens delar genom att påverka virtuell kraft på dem. 


En mekanisk del av maskinen är en degform. Figuren ~\ref{degfrom} visar en dagform som används för att knyta Raviolidegen manuellt genom att trycka på formens sidor. För detta projekt har planerats att en eller två motorer ska trycka degformens sidor. Olika tekniker för att överföra motorers rörelseenergi till Raviolimaskinens degform ska undersökas. 

\begin{figure}[h]
	\begin{center}
		\includegraphics[scale=0.08] {images/degform.jpg}
		\caption{Degform för manuell ifyllning}
		\label{degfrom}	
	\end{center}
\end{figure}

En annan teknik som ska användas på detta projekt är regleringsteknik. Reglering kommer vara användbar när det gäller att reglera t.ex. den strömmen som går till elektroniska komponenter. Regleringsteknik kan också användas för eventuella systemidentifiering. 

Mikrokontrollern som ska användas för detta projekt är en Arduino Due. Programmeringsspråket ska vara C, men det är tänkt att använda Arduino IDE i fall man inte hinner programmera med C.






\section{Hur uppgifterna specifieras}
Uppgifterna specificerar genom att dela upp projektet i tre stora delar. Det första delen är mekanik som består av maskinens formgivning och analys av alla krafter som kommer påverkas på varje del. Kravet på mekaniken specificerar genom att varje del av maskinen ska orka bära de krafter som kommer påverka det.

Vidare ska finnas elektronikdel som består av en krets för att strömförsörja motorer och några eventuella sensorer. Krav på elektroniken kan specificera genom att alla komponenter(motorer och eventuella sensorer) får tillräcklig ström för att fungera rätt. Det är också tänkt att utveckla strömregulator som ska reglera strömmen som går till de elektroniska komponenter. 

Programmeringsdel av projektet tar hand om timingen på ett sätt att olika komponenter fungerar rätt och i rätt tid. Programmerings uppgifter omfattas att läsa av sensorers värde och driva motorer. 


\section{Metodbeskrivning}
Raviolimaskinens delar kommer konstrueras med hjälp av en 3D-skrivare. Detta mest för att det blir mycket lättare att skapa vissa delar som är svårt om man vill bilda med metall. Det blir också billigare att printa delar med plast än bygga dem med t.ex. stål. Resursbehovet för att printa alla maskinens delar är självklart tillgång till en 3D-skrivare, 4 dagar i vecka för en månad.

Projektets elektronik kommer utvecklas med användning av några elektroniska komponenter. En prototyp av kretsen ska utvecklas och användas under projektet. Ett kretskort ska tillverkas när prototypen har fungerat som det ska. Resursbehovet för att tillverka ett kretskort ska vara tillgång till skolans elverkstad. 
Eftersom det är en egen ide, är det tänkt att använda skolan resurser liksom elverkstad eller 3D-skrivare. Nödvändiga Komponenter t.ex. ABS-filament till 3D-printern eller elektroniska komponenter ska skaffas av projektets deltagare.
\newpage



%\begin{itemize}
% \item Krav nr 1: Bygga en reglercentral som ska ersätta innedelen.
 %\item Krav nr 2: Reglercentralen ska kommunicera med utedelen enbart med RS-485 via kontakter F1 och F2.
% \item Krav nr 3: Reglercentralen ska hantera två insignaler. En analog signal mellan 0 till 10 volt och en digital signal.
% \item Krav nr 4: Reglercentralen ska översätta kompressorns frekvens så att tex 1 volt analog insignal motsvarar 10\% av kompressorn kapacitet, 2 volt 20 \% osv.
% \item Krav nr 5: Värme eller kylläge på utedelen ska väljas med den digitala insignalen till centralen.\\*
%\end{itemize}

\backmatter
% Referenser--------------------------------------------------------------------
\begin{thebibliography}{9}
\bibitem{engproc}
\href{http://www.wisegeek.com/what-is-ravioli.htm}{engproc}, engproc
\bibitem{max_485}
\href{https://www1.elfa.se/data1/wwwroot/assets/datasheets/jxMAX481E-491E_e.pdf}{MAX485}, Elfa
\bibitem{OSI}
\href{http://www.rejas.se/fritis/datorkommunikation/chap_protokoll.html}{OSI}, Rejas
\bibitem{Duplex}
\href{ http://en.wikipedia.org/wiki/Duplex_(telecommunications)}{Duplex}, Wikipedia
\bibitem{arduino}
\href{http://arduino.cc/en/Main/arduinoBoardDue}{Arduino due}, Arduino
\bibitem{RS och point-point mm}
\href{http://www.youtube.com/watch?v=2DQdEHvnqvI}{RS485}, Youtube
\bibitem{Om RS}
\href{http://www.ti.com/lit/an/slla070d/slla070d.pdf}{RS485}, Youtube
\bibitem{OSI modell}
\href{http://support.microsoft.com/kb/103884 }{RS485}, Youtube
\bibitem{CRC}
\href{http://en.wikipedia.org/wiki/Cyclic_redundancy_check }{CRC}, Wikipedia
\bibitem{Paritet}
\href{http://en.wikipedia.org/wiki/Parity_bit }{Paritet}, Wikipedia
\bibitem{Skillnader mellan RS korten}
\href{http://www.binarteknik.se/produkter/1201.phtml}{RS}, Wikipedia
\bibitem{RS485}
\href{http://en.wikipedia.org/wiki/RS-485}{RS485}, Wikipedia
\bibitem{Seriellt}
\href{http://www.rejas.se/fritis/datorkommunikation/chap_serpar.html}{Seriell}, Wikipedia
\iffalse
\bibitem{HALSALL} FRED HALSALL, {\it THRID EDITION DATA COMMUNICTIONS, COMPUTER NETWORKS AND OPEN SYSTEMS},Addsion wesley, Publishers (1994).
\bibitem{feynman} R. P. Feynman, {\it Phys. Rev.} {\bf 76}, 749 (1949); {\bf 76}, 769 (1949).
\fi
\end{thebibliography}

\end{document}
