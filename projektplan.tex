%\documentclass[11pt,a4paper,twoside]{book}
\documentclass[11pt,a4paper,oneside]{book}
 %\documentclass[oneside]{book}
% Misc

\usepackage[nottoc]{tocbibind} % Bibliography in toc
\usepackage{makeidx} 
\usepackage{graphicx}
\usepackage{color}
\usepackage{parskip} 
\definecolor{hhblue}{RGB}{0,73,133}

% Fancy headers
\usepackage{fancyhdr} \lhead{\slshape \nouppercase{\rightmark}}
\rhead{\slshape \nouppercase{\leftmark}}

% AMS Packages
\usepackage[intlimits]{amsmath} \usepackage{amscd}
\usepackage{amsxtra} \usepackage{amssymb}

% Hyperlinks
\usepackage{hyperref}
\hypersetup{colorlinks=true,linkcolor=hhblue,citecolor=hhblue,urlcolor=hhblue}

% Language
\usepackage[T1]{fontenc} \usepackage[swedish]{babel}
\usepackage[utf8]{inputenc}

\usepackage{titlesec}
\titleformat{\chapter}[display]
  {\normalfont\huge\bfseries}{}{0pt}{\Huge}
  %{\normalfont\huge\bfseries}{}{0pt}{\Huge}
  {\normalfont\huge\bfseries\color{hhblue}}
\titlespacing*{\chapter}
  {0pt}{10pt}{30pt} % changed
  
% Font
\usepackage[sc]{mathpazo} \usepackage{palatino}

% Page setup
\unitlength=1mm \oddsidemargin 0.0cm \evensidemargin 0.0cm \topskip
0.0cm \topmargin -0.5cm
% \headheight 0.0cm \headsep 0.0cm
\textheight 24.0cm \textwidth 16cm
\let\cleardoublepage\clearpage


\usepackage{float}
\restylefloat{table}

\setlength{\arraycolsep}{1.4pt}


\begin{document}
\pagestyle{empty}

\frontmatter

% Framsida---------------------------------------------------------------------
\begin{titlepage}
  \begin{center}
  \end{center}
  \vspace{3cm}
  \begin{center}
    \hrule \vspace{0.5cm}
    {\Huge \bfseries \sffamily \color{hhblue} Bakgrund(Raviolimaskin)}\\
   % \vspace{0.5cm} {\Large\emph{Kanske behövs någon undertext också}}
    \vspace{0.8cm} \hrule \vspace{2cm} {\Large{Reshad Ahmadi , Maryam Bayat}}\\
   %\hspace{1.5cm}
    %{\Large{Kalle Anka}}\\
    \vspace{2cm}
    \today\\
    \vspace{3cm}
    Examensarbete (Raviolimaskin)\\
    \vspace{1.5cm}
    Handledare: Kenneth Nilsson\\
   %\hspace{1.5cm}
    %{Tommy Salomonsson}\\ 
    \vspace{0.5cm} Examinator: Björn Åstrand \vfill
    \includegraphics[width=4cm]{images/hh_logo.jpg}\\
    HÖGSKOLAN I HALMSTAD\\
    Sektionen för Informationsvetenskap, \\
    Data- och Elektroteknik
  \end{center}
\end{titlepage}

% Sammanfattning----------------------------------------------------------------
%\chapter{Sammanfattning}
%\chapter{Abstract}
%Here is the nice abstract written in english.
% Förord-----------------------------------------------------------------------
%\chapter{Förord}
%\pagestyle{fancy}
% Innehållsförteckning---------------------------------------------------------
%\tableofcontents
%\mainmatter
%Here is the nice abstract written in english.
% Förord-----------------------------------------------------------------------
%\chapter{Förord}
%\pagestyle{fancy}


% Innehållsförteckning---------------------------------------------------------
\tableofcontents
\mainmatter

\chapter{Inledning}
Detta projekt ämnat till att skapa en Ravioli maskin. Raviolin är en traditionell italiensk maträtt bestående av rundor eller kvadratiska pastadeg med fyllning \cite{engproc}. Fyllningen kan bestå av till example köttfärs, skinka och ost. Raviolin serveras ofta i en tomatsås eller köttfärssås. vegetarisk ravioli kan exemplvis fyllas med purjiolök och spenat.\\

Att laga Ravioli hemma har varit jobbigt och tidskrävande. Det tar för mycket tid att fylla på en ravioli deg(utkavlade degen) och resultaten inte blir likadan för alla kuddar.\\

Det finns olika typer av Ravioli maskiner på markanden just nu. En typ av ravioli maskin som visas på figur1, underlättar processen men det mesta görs manuellt.\\

	 		 		\begin{figure}[h]
	 		 			\begin{center}
	 		 				\includegraphics[scale=0.5]{images/raviolimoldwithfilling.jpg}
	 		 				\caption{Raviolimaskin}
	 		 				\label{ravioli}	
	 		 			\end{center}
	 		 		\end{figure}
Den andra typen av maskinen är väldigt stor och priset är högt som medför att de inte kan användas av hushåll, se figur 3. Den typen finns färdig på marknaden.
 		\begin{figure}[h]
 			\begin{center}
 				\includegraphics[scale=2]{images/pastamachine.jpg}
 				\caption{Industriell Pasta/Raviolimaskin}
 				\label{pastamaskin}	
 			\end{center}
 		\end{figure}

Idén bakom projektet baseras på behov av en Ravioli maskin och potentiell marknad för den. Tanken är att man utvecklar en liten och bilig Ravioli maskin som kan vara användbar hemma.		

\section{Syfte och mål} % Goals
Detta projekt syftar på att utveckla en Raviolimaskin som ska fylla i Raviolidegen med ifyllningsmaterial och tillsluta degen automatiskt. Maskinen ska vara anpassad till hushåll i storlek, pris och användbarhet.

Följande krav har ställts på maskinen:

\begin{itemize}
	\item Maskinen ska automatiskt applicera fyllningsmaterialet på Raviolideg.
	\item Användaren ska kunna ha vilken fyllningsmaterial som helst för att fylla på raviolin.
	\item Maskinen ska automatisk tillsluta degen.
	\item Raviolideg ska placeras manuellt på maskinens degform.
	\item Raviolin ska plockas bort manuellt ur maskinen.
	\item Maskinen ska kommunicera med användaren via en display.
	\item Det ska vara lätt att rengöra maskinen.	
\end{itemize}

\section{Avgränsningar} % Constraints
Eftersom tiden är låst till en deadline som inte kan flyttas och personalresurer är begränsande,
kommer vi inte ha maskinen i metall.

En avgränsning ska vara att alla maskinens delar kommer att konstrueras med använndning av 3D-skrivare och plast som material. I slutet av projektet ska en plastmodel av maskinen utvecklas. Detta För att det är tar rätt mycket tid och resurser om man vill konstruera maskinen med t.ex. stål.


%-----------------------------------------------------------

\chapter{Backgrund}
\section{Existerande Raviolimaskiner}
De Raviolimaskiner som finns på marknad innehåller två huvuddelar, en pump för fyllning och en motordriven degform. 

Ett exempel på en Raviolimaskin visas på Figuren~\ref{raviolihemma}. Den består av två cylindriska degformar och en lucka där man fyller maskinen med fyllningsmaterial. Maskinens degfomar fungerar även som pump genom att de drar in fyllningsmaterialet när man snurrar dem m.h.a. ett handtag eller en motor.
 	\begin{figure}[h]
 		\begin{center}
 			\includegraphics[scale=0.4]{images/ravioli_machine_comment.jpg}
 			\caption{Raviolimaskin bestående av två cylindriska degformar~\cite{raviolimaskinbutik} }
 			\label{raviolihemma}	
 		\end{center}
 	\end{figure}
 
Den industeriella Raviolimaskinen som figur~\ref{pastamaskin} visar, fungerar med samma princip som maskinen på figur~\ref{raviolihemma}. Denna maskin gör allt från placering av en Raviolideg på maskinens degform till den slutar degen automatiskt.

Ett annat exempel på en industriell Raviolimaskin visas på figur ~\ref{industraviol_2}. Denna maskin består av en pump, en degform och en rullbana. Den typen av maskin fyller en Ravioli i tag och placering av Raviolideg på maskinen degform görs manuellt.
   .
 \begin{figure}[ht]
 	\begin{center}
 		\includegraphics[scale=0.4]{images/industriell_machine_comment.jpg}
 		\caption{Industriell Raviolimaskin som gör en Ravioli i tag(ref)}
 		\label{industraviol_2}	
 	\end{center}
 \end{figure}
 
 

 
\section{Teori}
Gruppmedlemmar har undersökt olika typer av pumpar för fyllning och olika sätt som degformen kan drivas med motor. Det finns olika modeller av pumpar och två av dem är Kolvdriven och kugghjul pump. För att driva degformen med motor analyserades två typer av motorer, likströmsmotor och stegmotor. De två typer av motorer lämpar detta projekt p.g.a. de är lätt att styra och tidigare erfarenhet att använda dem i ett projekt.

En undersökning gjordes på hur man kan detektera när degformen har pressat nog Raviolidegen för att tillsluta det med tillämpning av likströmsmotor eller stegmotor.

\subsection{Olika typer av pumpar}
\textbf{Kolvpump}\\
En kolvpump pumpar ingredienserna med hjälp av en kolv som rör sig fram och tillbaka i en cylinder. Pumpen är utformad för att hantera vätskor, halvfasta och trögflytande produkter och den häller ingredienserna på degen med hjälp av ett munstycke. Doseringsvolymen på matrialet kan bestämmas genom att helt enkelt öka eller minska kolvens rörelse. Figuren~\ref{kolvpump} visar example på en kolvpump.

Födelar med pumptekniken:
\begin{itemize}
	\item Påfyllningsvolymen är exakt doserad för att minska slöseriet.
	\item Fördelningen av olika produkter och halvfasta ämnen i samma behållare är korrekt repeterbar.
	\item Pumptekniken mäter ingredienserna med precision, tack vare servodrivenkolv.
	\item Pumpen kan rengöras på plats utan nedmontering.
\end{itemize}

\begin{figure}[h]
	\begin{center}
		\includegraphics[scale=0.5]{images/maxresdefault.jpg}
		\caption{Kolvpump med exakt dosering för fyllningsmaterial~\cite{Dosering pump}}
		\label{kolvpump}	
	\end{center}
\end{figure}

\newpage
\textbf{Kugghjulspump}\\*
Figuren~\ref{kugghjulpump} visar en kugghjulspump som består av två kugghjul, ett drivande och ett drivet kugghjul.  Materialet följer luckorna mellan kuggarna genom pumpen. Kugghjulspumpar lämpar sig bäst för höga pumphöjder(vertikala sträcka mellan slutväxel och pump)\cite{kugghjul pump}.

Pumpen används allmänt i moderna hydrauliska system på grund av höga prestanda och lång livslängd.

\begin{figure}[h]
	\begin{center}
		\includegraphics[scale=0.25]{images/68637(1).jpg}
		\caption{Kugghjulspump, ett av hjulen drivs av den andra}
		\label{kugghjulpump}	
	\end{center}
\end{figure} 
\newpage
\subsection{Motordriven degform}
Figuren  ~\ref{degform} Visar en degform för manuell fyllning av en Ravioli. Den fungerar genom att man lägger Raviolidegen på formen, efter detta läggs fyllningsmaterial på degen och sist tillsluter man de-gen genom att pressa formens handtag mot varandra. Degformen kan sluta degen automatiskt med hjälp av motorer. Lämpliga typer av motorer  är likströms- eller stegmotor.
\begin{figure}[h]
	\begin{center}
		\includegraphics[scale=0.75]{images/ravioli_mould_trimed_1.jpg}
		\caption{Degform}
		\label{degform}
	\end{center}
\end{figure}


\textbf{Likströmsmotor}\\*
Likströmsmotorer är den vanligaste motor som sitter i många olika produkter som leksaker, dataspel mm\cite{likstromsmotor}. Strömmen som en likströmsmotor förbrukar beror på belastningen. Denna egenskap kan användas som en sensor för att identifiera t.ex. hinder och i detta fall när degformen har pressat Raviolidegen nog för att tillsluta den.\\

\textbf{Stegmotor}\\*
Den här typen av motor liknar likströmsmotor men den skiljer sig från likströmsmo-tor genom en unik egenskap: stegmotor roterar et steg vid en strömpuls. Steget minskar med ökat antal poler i statorn. Genom att beräkna antal pulsar som skickas till steg-motorn, kan man exakt positionera ett objekt \cite{stegmotor}.

\subsection{Styrenhet}
För att Raviolimaskinen ska fungera krävs en styrenhet. Styrenheten ska kunna driva motorer för att öppna och stänga degformen samt att kunna pumpa fram fyllningsmaterialet.

\textbf{Maskinstyrning}\\
Maskinstyrning delas in  i två kategorier, centralstyrning och sekvensstyrning.Vid centralstyrning eller tidsbaserade styrning ges order i tidsföljd utan krav på kvittering. Andra typen är sekvensstyrning, där varje nytt steg initieras av kvittering som anger  att order i föregående steget blivit utförd.

Raviolimaskinen styrs genom en sekvensstyrning, där t.ex. ska fyllningsmaterial pumpas fram på degen innan formen tillslutas. Nedan jämförs olika modeller av styrenheter som passar för sekvensstyrning.

\textbf{Arduino due}\\
Arduino är en plattform baserad på öppen källkod och hårdvara. Den består av en  programmerbar kretskort (mikro) och programvara. På kortet finns analoga och digitala I/O pinnar och  PWM (Pulse-Width Modulation)för drivning av motorer. Arduino Due kan interagera med knappar, lysdioder, motorer, högtalare, GPS-enheter, kameror och internet. Det finns okcså inbyggt stöd för LCD, I2C, SPI och timers.

Utvecklingsmjlön för Arduino är Arduino IDE (Integrated Development Environment) som körs på  dator. Den används för att skriva och ladda upp programkod till plattformen \cite{Arduino1}. Man programmerar i Arduinos egen miljö och programmeringsspråket bygger på  C/C++ där man har många färdiga rutiner som gör det enkelt att programmera\cite{Arduino2}. Editorn fungerar dessutom i alla operativsystem (Windows, MacOS och Linux). 
 
\textbf{Raspberry pi}\\
Raspberry pi är enkortsdator som ansluts till en datorskärm, och använder ett vanligt tangentbord och mus. Den  används som en vanlig dator och  är kapabel att göra allt som en stationär dator gör. Raspberry pi har inget inbyggt minne för operativsystemet och filer, istället används ett externt SD-kort för fillagring. På kortet sitter 26 stycken pinnar som kallas för GPIO(General purpose input/output) . Några av dessa pinnar har extra funktioner såsom en I2C-buss, SPI buss och UART seriella anslutningar\cite{Raspberry1}.

Raspberry pi är en Linux-baseratoprerativsystem och det primära programmeringspråket är Scratch och Python, men det är möglighet att programmera i C/C++.

\textbf{PLC (Programmable Logical Controller) }\\
PLC är en programmerbar dator för styrning av industriella maskiner/processer. Detta består av en centralenhet, ett minne, ingångar som kan ta elektriska signaler och utgångar som sänder ut signaler. Centralenheten läser givare genom ingångskortet, styr systemet genom de givna instruktionerna och reglerar utsignalerna därefter. Användarområderna  är främst till automation (Mekanisering) inom industri och hissanordningar.

\subsection{Kommunikation med användaren}
Det finns inget specillet interface för både industriell och manuel Raviolimaskin. De flesta av maskiner använder sig av några knappar som start/stopp knapp. Här beskrivs två alternativa lösningar som man kan använda för kommunikation mellan Raviolimaskinen och användaren.\\

\textbf{Kommunikation via dator}\\
Kommunikation sker mellan en dator och maskinens styrenhet via USB. Styrenheten skickar information om maskinens status till datorn seriellt\cite{Arduinocookbook}. Datorn tar emot de inkomna seriella signaler, tolkar dem och visar information på skärmen på ett läsbart sätt för en användare.\\


\textbf{Display}\\
LCD används i olika typer av enheter. Fördelar med att använda LCD som interface är effektivitet, storlek och låg kostnad.  LCD kan kopplas direkt till plattformen, vilket är bra för maskinen som inte är stor i storleken.



\chapter{Metod}
Raviolimaskin som ska tas fram är tänkt att fylla en Ravioli i tag. Detta görs genom att utforma maskinens degform så att endast en Raviolideg kan placeras på formen. Man ska placera en Raviolideg på maskinens degform. Efter detta ska degformen hissa upp till maskinens pump. Detta görs för att minska avståndet mellan degform och maskinens pump som i sin tur minskar materialslöseriet. I näst ska fyllningsmaterialet pumpas på Raviolidegen som är placerad på formen. Till slut hissas ner degformen och sluter till degformen. Figuren~\ref{blockschema} maskinens blockschema .

\begin{figure}[ht]
	\begin{center}
		\includegraphics[scale=0.8]{images/blockschema.png}
		\caption{Industriell Raviolimaskin som gör en Ravioli i tag(ref)}
		\label{blockschema}	
	\end{center}
\end{figure}

\section{Maskinens pump för fyllning}
Val av pumpen görs med tanke på Raviolimaskinens behov och tillverknings möjligheten. Kolvpump kan pumpa fyllningsmaterialet med högre noggrannhet än kugghjulspumpar.  Däremot blir det mindre kostnad att ta fram en kugghjulspumps prototyp med tanke på att det behövs en motor för kugghjuls pump i jämförelse med två motorer för kolvpump.
För detta projekt ska en variant av kugghjuls pump tillverkas som består av endast ett kugghjul. Pumpen ska tillämpas för Raviolimaskinen genom att implementera ett filter i pumpen som ska filtrera eventuell vätska i fyllningsmaterialet. Pumpens kugghjul drivs med användning av en stegmotor som kan positionera pumpens kugghjul exakt i den önskade positionen.

\section{Motordriven degform}
Maskinens degform ska stänga en Raviolideg med hjälp av två likströmsmotorer. Motorers rotationsenergi ska överföras till degformen med användning av en kugghjulsväxel. Kugghjulsväxel ska bestå av ett kugghjul som sitter på en likströmsmotor och ett kugghjul som ska monteras på baksidan av masinens degform.

Utväxling av kugghjulsväxel räknas med formel:
\begin{center}
\LARGE \textbf{	$ i = \frac{N1}{N2}$}
\end{center}
där i är utväxlingen, N1 är det drivande kugghjulets varvtal och N2 är det drivna kugghjulets varvtal. 

Det är också tänkt att implementera möjligheten att hissa upp och ner maskinens degform. Detta görs för att minska avstånd mellan maskinens degform och pump. För att genomföra detta ska en kuggväxel bestående av två kugghjul och två kuggsträngar användas, se figur.~\ref{maskinens_baksida_metod}

\begin{figure}[ht]
	\begin{center}
		\includegraphics[scale=0.8]{images/maskinBaksida.jpg}
		\caption{Maskinens baksida som innehåller kuggsträngar och kugghjul(kuggväxel)}
		\label{maskinens_baksida_metod}	
	\end{center}
\end{figure}

\section{Design av maskin}
Maskinens alla delar ritas med användning av ”SolidWorks” som är ett CAD program. Utöver möjligheten att rita och designa möjliggör SolidWorks simulering av de ritade delarna. Med denna egenskap kan man kontrollera och se hur maskinenes olika delar fungerar ihop som en maskin. På programmets simulator kan man påverka en visuell kraft på en del och se hur den reagerar innan man bygger en del.
\begin{figure}[ht]
	\begin{center}
		\includegraphics[scale=0.8]{images/hissEdited.png}
		\caption{Industriell Raviolimaskin som gör en Ravioli i tag(ref)}
		\label{simulering}	
	\end{center}
\end{figure}
\section{Tillverknings av maskinens delar}
Maskinens delar är tänkt att printas med en 3D-skrivare. Fördelen med 3D-printern är att det är enkelt att printa en del som är svårt att tillverka med hand med minst fel. Det blir också mycket snabbare att printa med 3D-skrivare i jämförelse med tillverkning av samma del med hand i en verkstad. 
\section{Sensor}
I detta projekt används tre olika metoder för att identifiera motorers position och för säkerhet av maskinen.
\subsection{Strömavläsning}
Strömmen som går genom en likströmsmotor ökar med ökad belastning. Ett exempel är motorer som ska hissa upp och ner maskinens degform, se figur~\ref{maskinens_baksida_metod}. När de har hissat upp degformen till ändläge på kuggsträngar, ökar strömmen i motorer som kan avläsas för att identifiera motorers position.
\subsection{IR-sensor}
Det ska tas fram en IR-sensor genom att använda en IR-sändare som skickar ut IR-signaler och en fototransistor som tar emot IR signaler. Under tiden som fototransistor tar emot signaler definieras som normalläge. Så fort som fototransistor inte tar IR-signaler är då ett objekt mellan sensorer.


\section{Styrenhet}
Enligt kravspecifikation jämfördes de tre olika alternativen för att välja på vilken processor lämpar sig bäst för detta projekt.I första steget bedömdes alla alternative i enlighet med att styrenheten ska kunna ha  mögligheten att styra Raviolimaskinen geneom sekvensstyrning, vilken är en viktig del i projektet.Efter jämförlse mellan PLC och de både mikrokontroller Arduiono och Rasberry pi bestämdes att använda en mikrokontroller med tanke på priset och tidigare erfanhet. \\

PLC är ett system för automationteknik, som används mest för styrning och reglering för industriella processor. Däremot används mikrokontroller Arduiono och Rasberry pi för inbyggda system.

Gemensama fördelar mellan Arduino och Rasberry Pi
\begin{itemize}
	\item Gott om anslutningsmögligheter(både analogt och digitalt).
	\item Den är relativt billig samt är enkel att programmera.
	\item Bra för styrning av många motorer genom PWM
	\item Möglighet för sekvensstyrning.
\end{itemize}

 Arduino kräver inte specillet operativsystem medans Rasberry pi operativsytem är specillet.Fördelen med  Arduino är att det har inbyggt minne men Rasberry pi använder sig av ett extern SD kort. Arduino har realtid och analog funktioner som Rassbery inte har. Pi är inte så flexibel tex att läsa analoga sensorer kräver extra hårdvara stöd.


%\section{Blockschema}
%\section{Testning}

\chapter{Hittlis resultat}
\section{Maskinens design}
Ritning av maskinens delar är 90 \% klara. Det återstår småändringar som kan förekomma under tillverknings av delarna. Figuren~\ref{helmaskin} visar en översikt av maskinens ritning på SolidWorks.

\begin{figure}[h]
	\begin{center}
		\includegraphics[scale=0.75]{images/hela_maskin.png}
		\caption{Översikt av hela maskinen bestående av pump, degform och hiss.}
		\label{helmaskin}
	\end{center}
\end{figure}


\begin{figure}[h]
	\begin{center}
		\includegraphics[scale=0.75]{images/maskinBaksida.jpg}
		\caption{Maskinenes baksida bestående av två kuggsträngar och två kugghjul som sitter på två motorer.}
		\label{helmaskinbaksida}
	\end{center}
\end{figure}

%\chapter{Tidsplan}
%Examenarbetet krävs 20 timmar arbetsinsats i veckan. Därför måste läggas minst 350 timmar 
för att kunna klara arbetet. Följande är en grovplanering till projektet med tanke på de olika uppgifter som ska göras. \\


\begin{figure}[h]
	\begin{center}
		\includegraphics[scale=0.5] {images/tidsplan.png}
		\label{Tidsplan}	
	\end{center}
\end{figure}
\section{kommunikation och styrning}
Prototypen består av motorer, kugghjular, sensorer och mikrokontroller. Centralt i det systemet bestämdes att använda Arduino due som ska kommunicera med alla elektriska komponenter.

Anledningen för valet av Arduino due är att den fyller specifikationkraven som ställdes för styrenheten. Utöver dessa finns tillräcklig antal portar för att koppla motorer och sensorer samt att det kan bestämmas farten på hissen som går upp och ner genom PWM signalen. Arduino är en liten plattform, vilken är bra för maskinens begränssad storlek. Tidigare erfanhet att använda och programmera med Arduino samt mögligheten att programmera i C/C++ var också en fördel.

För att kunna kommunicera med användaren används en display som heter 3.2" TFT LCD Touch shield, vilket hjälper användaren att övervaka maskinens tillstånd och eventuella fel.

Figuren~\ref{flodesschma} visar systemets flödesschema.

\begin{figure}[ht]
	\begin{center}
		\includegraphics[scale=1.7]{images/Flowchart.png}
		\caption{Systemets flödesschema.}
		\label{flodesschma}	
	\end{center}
\end{figure}
%\section{Design}


%\begin{itemize}
% \item Krav nr 1: Bygga en reglercentral som ska ersätta innedelen.
 %\item Krav nr 2: Reglercentralen ska kommunicera med utedelen enbart med RS-485 via kontakter F1 och F2.
% \item Krav nr 3: Reglercentralen ska hantera två insignaler. En analog signal mellan 0 till 10 volt och en digital signal.
% \item Krav nr 4: Reglercentralen ska översätta kompressorns frekvens så att tex 1 volt analog insignal motsvarar 10\% av kompressorn kapacitet, 2 volt 20 \% osv.
% \item Krav nr 5: Värme eller kylläge på utedelen ska väljas med den digitala insignalen till centralen.\\*
%\end{itemize}

\backmatter
% Referenser--------------------------------------------------------------------
 \bibliographystyle{99}
\begin{thebibliography}{1}
\bibitem{raviolimaskinbutik}
\href{http://www.kitchenaid.com/shop/-[KRAV]-400107/KRAV/}{http://www.kitchenaid.com/shop/-[KRAV]-400107/KRAV}, kitchenAid	

\bibitem{Dosering pump}
\href{http://gb.pcm.eu/en/food-applications/filling-dosys-technology.html}{http://gb.pcm.eu/en/food-applications/filling-dosys-technology.html},Dosering pump

\bibitem{kugghjul pump}
\href{http://hj.diva-portal.org/smash/get/diva2:219806/FULLTEXT01.pdf}{http://hj.diva-portal.org/smash/get/diva2:219806/FULLTEXT01.pdf},kugghjul pump

\bibitem{engproc}
\href{http://www.wisegeek.com/what-is-ravioli.htm}{http://www.wisegeek.com/what-is-ravioli.htm}, Engproc
\bibitem{likstromsmotor}
\href{http://www.drivteknik.nu/skolan/motor/stegmotor}{http://www.drivteknik.nu/skolan/motor/stegmotor}, Likströmsmotor
\bibitem{stegmotor}
\href{http://www.ne.se.ezproxy.bib.hh.se/uppslagsverk/encyklopedi/l\%C3\%A5ng/stegmotor}{http://www.ne.se.ezproxy.bib.hh.se/uppslagsverk/encyklopedi/l\%C3\%A5ng/stegmotor}, Stegmotor
%\href{http://www.lawicel-shop.se/dept/Arduino_74952/SWE/SEK}{http://www.lawicel-shop.se/dept/Arduino_74952/SWE/SEK},Arduino
\href{https://learn.sparkfun.com/tutorials/what-is-an-arduino}{https://learn.sparkfun.com/tutorials/what-is-an-arduino },Arduino1
%\href{http://www.lawicel-shop.se/dept/Arduino_74952/SWE/SEK}{http://www.lawicel-shop.se/dept/Arduino_74952/SWE/SEK},Arduino2
\bibitem{Arduino3}
\href{https://learn.sparkfun.com/tutorials/what-is-an-arduino}{https://learn.sparkfun.com/tutorials/what-is-an-arduino},Arduino3
\bibitem{Raspberry}
\href{https://www.raspberrypi.org/help/what-is-a-raspberry-pi/}{https://www.raspberrypi.org/help/what-is-a-raspberry-pi/},Raspberry
\bibitem{Raspberry1}
\href{http://computers.tutsplus.com/tutorials/controlling-dc-motors-using-python-with-a-raspberry-pi--cms-20051}{http://computers.tutsplus.com/tutorials/controlling-dc-motors-using-python-with-a-raspberry-pi--cms-20051}, Raspberry1





\end{thebibliography}


\end{document}
