Figuren  ~\ref{degform} Visar en degform för manuell fyllning av en Ravioli. Den fungerar genom att man lägger Raviolidegen på formen, efter detta läggs fyllningsmaterial på degen och sist tillsluter man de-gen genom att pressa formens handtag mot varandra. Degformen kan sluta degen automatiskt med hjälp av motorer. Lämpliga typer av motorer  är likströms- eller stegmotor.
\begin{figure}[h]
	\begin{center}
		\includegraphics[scale=0.75]{images/ravioli_mould_trimed_1.jpg}
		\caption{Degform}
		\label{degform}
	\end{center}
\end{figure}


\textbf{Likströmsmotor}\\*
Likströmsmotorer är den vanligaste motor som sitter i många olika produkter som leksaker, dataspel mm\cite{likstromsmotor}. Strömmen som en likströmsmotor förbrukar beror på belastningen. Denna egenskap kan användas som en sensor för att identifiera t.ex. hinder och i detta fall när degformen har pressat Raviolidegen nog för att tillsluta den.\\

\textbf{Stegmotor}\\*
Den här typen av motor liknar likströmsmotor men den skiljer sig från likströmsmo-tor genom en unik egenskap: stegmotor roterar et steg vid en strömpuls. Steget minskar med ökat antal poler i statorn. Genom att beräkna antal pulsar som skickas till steg-motorn, kan man exakt positionera ett objekt \cite{stegmotor}.
