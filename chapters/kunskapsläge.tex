Beskrivning av kunskapsläget (detta är ett obligatoriskt moment i din rapport). Inom vilket område skall du redogöra för kunskapsläget? Om ditt projekt t.ex. handlar om att ta fram ett gränssnitt kan du välja att redogöra för kunskapsläget inom användargränssnitt – vad kännetecknar ett bra gränssnitt, vilka tester/undersökningar har gjorts, etc. Om ditt projekt t.ex. handlar om att implementera en specifik sak i VHDL är det lämpligt att du redogör för varför VHDL är speciellt lämpligt att använda, vilka andra alternativ som finns och fördelar samt nackdelar för olika alternativ. (Denna del, ”beskrivning av kunskapsläget”, är ett moment som högskolan kräver i projektet som ett prov på er förmåga att inhämta och sammanställa kunskap.)