Raviolimaskinens delar kommer konstrueras med hjälp av en 3D-skrivare. Detta mest för att det blir mycket lättare att skapa vissa delar som är svårt om man vill bilda dem med metal. Det blir också billigare att printa delar med plast än bygga dem med t.ex. stål. Resursbehovet för att printa alla de delar är självklart tillgång till en 3D-skrivare, 4 dagar i vecka för en månad.\\

Projektets elektronik kommer utvecklas med användning av några elektroniska komponenter. En prototyp av kretsen ska utvecklas och användas under projektet. Ett kretskort kommer tillverkas när prototypen har fungerad som det ska. Resursbehovet för elektroniken utöver komponenterna ska möjligen vara tillgång till elverkstad för att kunna ett kretskort.\\

Eftersom det är en egen ide, är det tänkt att använda skolan resurser liksom elverkstad eller 3D-skrivare. Nodvändiga Komponenter kommer liksom ABS-filament till 3D-printern eller elektronikska komponenter ska skaffas av projektets deltagare.

