Raviolimaskinens delar kommer konstrueras med hjälp av en 3D-skrivare. Detta mest för att det blir mycket lättare att skapa vissa delar som är svårt om man vill bilda med metall. Det blir också billigare att printa delar med plast än bygga dem med t.ex. stål. Resursbehovet för att printa alla maskinens delar är självklart tillgång till en 3D-skrivare, 4 dagar i vecka för en månad.

Projektets elektronik kommer utvecklas med användning av några elektroniska komponenter. En prototyp av kretsen ska utvecklas och användas under projektet. Ett kretskort ska tillverkas när prototypen har fungerat som det ska. Resursbehovet för att tillverka ett kretskort ska vara tillgång till skolans elverkstad. 
Eftersom det är en egen ide, är det tänkt att använda skolan resurser liksom elverkstad eller 3D-skrivare. Nödvändiga Komponenter t.ex. ABS-filament till 3D-printern eller elektroniska komponenter ska skaffas av projektets deltagare.