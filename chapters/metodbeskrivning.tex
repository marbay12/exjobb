Raviolimaskinens delar kommer konstrueras med hjälp av 3D-skrivare. Detta mest för att det blir mycket lättare att skapa vissa delar som är svårt om man vill anlägga med metal. Det blir också billigare och printa ut alla delar med plast än att bygga dem med t.ex. stål. Resursbehov för att printa alla de delar är självklart tillgång till en 3D-skrivare, 4 dagar i vecka på en månad för att hinna med allt.\\

Projektets kretskort kommer utvecklas med användning av några elektroniska komponenter. Resursbehovet för elektroniken utöver komponenterna ska möjligen vara tillgång till elverkstad för att kunna tillverka ett kretskort.\\

