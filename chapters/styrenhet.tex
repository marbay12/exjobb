Enligt kravspecifikation jämfördes de tre olika alternativen för att välja på vilken processor lämpar sig bäst för detta projekt.I första steget bedömdes alla alternative i enlighet med att styrenheten ska kunna ha  mögligheten att styra Raviolimaskinen geneom sekvensstyrning, vilken är en viktig del i projektet.\\


PLC är ett system för automationteknik, som används mest för styrning och reglering för industriella processor. Däremot används mikrokontroller Arduiono och Raspberry pi för inbyggda system. Det som är bra med Arduino och Raspberry pi jämfört med PLC är att kostnadpriset för de är inte så mycket, vilket är det viktig för detta ska vara en hemanvändare maskin.\\

Gemensama fördelar mellan Arduino och Raspberry pi
\begin{itemize}
	\item Gott om anslutningsmögligheter(både analogt och digitalt).
	\item Den är relativt billig samt är enkel att programmera.
	\item Bra för styrning av många motorer genom PWM.
	\item Möglighet för sekvensstyrning.
\end{itemize}

 Arduino kräver inte specillet operativsystem medans Raspberry pi operativsytem är baserad på Linux.Fördelen med  Arduino är att det har inbyggt minne men Rasberry pi använder sig av ett extern SD kort. Arduino har realtid och analog funktioner som Rassbery inte har. Pi är inte så flexibel tex att läsa analoga sensorer kräver extra hårdvara stöd.

