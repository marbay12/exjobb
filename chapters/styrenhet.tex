Enligt kravspecifikation jämfördes de tre olika alternativen för att välja på vilken processor lämpar sig bäst för detta projekt.I första steget bedömdes alla alternative i enlighet med att styrenheten ska kunna ha  mögligheten att styra Raviolimaskinen geneom sekvensstyrning, vilken är en viktig del i projektet.Efter jämförlse mellan PLC och de både mikrokontroller Arduiono och Rasberry pi bestämdes att använda en mikrokontroller med tanke på priset och tidigare erfanhet. \\

PLC är ett system för automationteknik, som används mest för styrning och reglering för industriella processor. Däremot används mikrokontroller Arduiono och Rasberry pi för inbyggda system.

Gemensama fördelar mellan Arduino och Rasberry Pi
\begin{itemize}
	\item Gott om anslutningsmögligheter(både analogt och digitalt).
	\item Den är relativt billig samt är enkel att programmera.
	\item Bra för styrning av många motorer genom PWM
	\item Möglighet för sekvensstyrning.
\end{itemize}

 Arduino kräver inte specillet operativsystem medans Rasberry pi operativsytem är specillet.Fördelen med  Arduino är att det har inbyggt minne men Rasberry pi använder sig av ett extern SD kort. Arduino har realtid och analog funktioner som Rassbery inte har. Pi är inte så flexibel tex att läsa analoga sensorer kräver extra hårdvara stöd.

