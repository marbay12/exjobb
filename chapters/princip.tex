Raviolimaskin som ska utvecklas är tänkt att fylla en Ravioli i tag. Detta görs genom att utforma maskinens degform så att endast en Raviolideg kan placeras på formen. Efter detta ska degformen hissas upp till maskinens pump. Detta görs för att minska avståndet mellan degform och maskinens pump som i sin tur minskar materialslöseriet. Näst ska fyllningsmaterialet pumpas på Raviolidegen som är placerad på formen. Till slut hissas ner degformen och sluter till Raviolidegen. Figuren~\ref{blockschema} visar maskinens olika tillstånd.

\begin{figure}[ht]
	\begin{center}
		\includegraphics[scale=0.8]{images/blockschema.png}
		\caption{Maskinens olika tillstånd}
		\label{blockschema}	
	\end{center}
\end{figure}
